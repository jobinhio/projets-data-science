\documentclass[12pt]{article}
\usepackage{a4wide}
\usepackage{amsmath,amssymb}
\usepackage{bm}
\usepackage[colorlinks]{hyperref}
\usepackage{listings}
\usepackage{graphicx}







\begin{document}

\title{Report V1}
\author{}
\maketitle



\tableofcontents

\section{Introduction}

Fonctions et tâches :
Comprendre la méthodes des bases réduites
Implémenter un exemple en 1d en Python et/ou Feel++ avec la méthodes des éléments finis
Entrainer un modèle apprenant l'erreur de projection à l'aide de pyTorch/Tensorflow
Extension au cas 2D
Compétences :
Implémenter une méthodes d'éléments finis, étude de convergence
Implémenter une méthode d'apprentissage à l'aide de réseaux de neurones profonds
écrire un rapport détaillé et préparer une communication orale en vue d'une soutenance



\section{Méthode des Elements finis}

\subsection{Expliquation de la methode}


\subsection{Etude de convergance}


\section{Méthode des Bases reduites}

\subsection{Expliquation de la methode}


\subsection{Etude du conditionnement ou sélection de la dimension}

\subsection{Qualité de la Base Reduite}

\subsection{Etude de l'erreur }


\section{Conclusion}


\begin{thebibliography}{9}


\bibitem{texbook}
Donald E. Knuth (1986) \emph{The \TeX{} Book}, Addison-Wesley Professional.

\bibitem{lamport94}
Leslie Lamport (1994) \emph{\LaTeX: a document preparation system}, Addison
Wesley, Massachusetts, 2nd ed.


\end{thebibliography}

\end{document}
